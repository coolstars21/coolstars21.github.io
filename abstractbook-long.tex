
\documentclass[11pt,twoside, openright]{report}
\usepackage{latexsym}
\usepackage[twoside,bindingoffset=1cm,inner=2cm,outer=2cm,bottom=3cm,top=2.5cm,headsep=1.0cm]{geometry}
\usepackage{amsmath}
\usepackage[utf8]{inputenc}
\usepackage{graphicx}
\usepackage{hyperref}
\usepackage[T1]{fontenc}
%\usepackage{showframe}

\usepackage{fancyhdr}
\setlength{\headheight}{15pt}

\pagestyle{fancy}

\fancyhf{}
\fancyhead[LE,RO]{\thepage}
\fancyhead[RE]{\textit{ \nouppercase{ \leftmark } } }
\fancyhead[LO]{\textit{ Cool Stars 20 abstract booklet } }
\fancyfoot[C]{last updated: \today}

\fancypagestyle{plain}{
  \fancyhf{}
  \renewcommand{\headrulewidth}{0pt}
  \renewcommand{\footrulewidth}{0pt}
  \fancyfoot[C]{last updated: \today}
}

\begin{document}
\begin{titlepage}
  \centering
  \includegraphics[width=.6\textwidth]{Logoround-04}
  \vspace{2cm}
  \par{\scshape\LARGE $20^{th}$ Cambridge Workshops of Cool Stars, Stellar Systems and the Sun \par}
  \vspace{1cm}{\scshape\Large July 29 - Aug 3 2018, Boston / Cambridge, USA \par}
  \vspace{1.5cm}
  {\huge\bfseries Abstract booklet\par}
  \vfill

        % Bottom of the page
  {\large Last updated: \today\par}
\end{titlepage}

\tableofcontents

\chapter{Talks}
        
          \section[Clément Baruteau: MWC 758] { MWC 758 }
\textbf{ Clément Baruteau}\footnote{\href{mailto: clement.baruteau@free.fr}{clement.baruteau@free.fr} };  Marcelo Barazza\\
    (1) IRAP; (2)  IPAG\\


    
    \noindent
    science theme: Stars\\
    \emph{schedule: Mon, 8:45 - 9:15  }\\


 
    
\vspace{3 mm}

      Resolved ALMA and VLA observations indicate the existence of two dust traps in the protoplanetary disc MWC 758. By means of 2D gas+dust hydrodynamical simulations post-processed with 3D dust radiative transfer calculations, we show that the spirals in scattered light, the eccentric, asymmetric ring and the crescent-shaped structure in the (sub)millimetre can all be caused by two giant planets: a 1.5-Jupiter mass planet at 35 au (inside the spirals) and a 5-Jupiter mass planet at 140 au (outside the spirals). The outer planet forms a dust-trapping vortex at the inner edge of its gap (at ∼85 au), and the continuum emission of this dust trap reproduces the ALMA and VLA observations well. The outer planet triggers several spiral arms which are similar to those observed in polarised scattered light. The inner planet also forms a vortex at the outer edge of its gap (at ∼50 au), but it decays faster than the vortex induced by the outer planet, as a result of the disc’s turbulent viscosity. The vortex decay can explain the eccentric inner ring seen with ALMA as well as the low signal and larger azimuthal spread of this dust trap in VLA observations. Finding the thermal and kinematic signatures of both giant planets could verify the proposed scenario. 

        
          \section[Sébastien Deheuvels: Red Giant] { Red Giant }
\textbf{ Sébastien Deheuvels}\footnote{\href{mailto: clement.baruteau@free.fr}{clement.baruteau@free.fr} };  Anthony Noll;  François Lignières;  Jérôme Ballot\\
    (1) IRAP; (2)  IPAG; (3)  IRAP; (4)  UCL\\


    
    \noindent
    science theme: Planets\\
    \emph{schedule: Tue, 10:30 - 11:00  }\\


 
    
\vspace{3 mm}

      Whatever its trigger, the RWI leads to the formation of one or several vortices, which tend to merge and form a single large-scale anticyclonic vortex. An anticyclonic vortex forms a patch of closed elliptical streamlines about a local pressure maximum. The vortex flow tends to maintain dust on the same elliptical streamlines, gas drag tends to drive dust to- wards the vortex centre, while dust turbulent diffusion tends to spread it out (Chavanis 2000; Youdin 2010; Lyra \& Lin 2013). In addition, the vortex’s self-gravity causes dust par- ticles to describe horseshoe U-turns relative to the vortex centre, much like in the circular restricted three-body prob- lem, despite the vortex not being a point mass (Baruteau \& Zhu 2016). The competition between the aforementioned ef- fects implies that dust particles of increasing size get trapped farther ahead of the vortex centre in the azimuthal direc- tion (Baruteau \& Zhu 2016). Vortices triggered by the RWI could play a key role in planet formation by slowing down or stalling the dust’s inward drift due to gas drag, while po- tentially allowing dust to grow to planetesimal sizes or even planetary sizes (Lyra et al. 2009; S ́andor et al. 2011). 

        
          \section[Jérôme Ballot: Heliosismo] { Heliosismo }
\textbf{ Jérôme Ballot}\footnote{\href{mailto: clement.baruteau@free.fr}{clement.baruteau@free.fr} };  Stéphane Charpinet;  Michel Rieutord;  Quidonc\\
    (1) IRAP; (2)  UCLA; (3)  UCSC; (4)  CEA\\


    
    \noindent
    science theme: Disks\\
    \emph{schedule: Wed, 11:15 - 11:30  (invited)  }\\


 
    
\vspace{3 mm}

      nsimu(s)ds represents the number of super-particles in the size interval [s, s + ds] in the simulation). This par- ticular scaling of the dust’s size distribution is chosen for computational reasons, as it implies that there is approxi- mately the same number of particles per decade of size. An important note is that the radiative transfer calculations do need a realistic size distribution for the dust, but the only input that they need from the hydrodynamical simulations is the spatial distribution of the dust particles. This is the reason why we can choose any size distribution in the sim- ulations, as long as there is e 

        
          \section[Laurène Jouve: MRI in stars] { MRI in stars }
\textbf{ Laurène Jouve}\footnote{\href{mailto: clement.baruteau@free.fr}{clement.baruteau@free.fr} }\\
    IRAP\\


    
    \noindent
    science theme: Stars\\
    \emph{schedule: Thu, 14:30 - 15:15  }\\


 
    
\vspace{3 mm}

      Turbulent transport of angular momentum is modelled by a constant alpha turbulent viscosity, α = 10−4. This rather low level of turbulence is representative of the discs midplane from a few tens to a hundred au, as suggested by observations of the dust continuum in the submillimetre (see, e.g., Pinte et al. 2016, for the modelling of the HL Tau disc), and according to 3D, non-ideal, local magnetohydro- dynamic (MHD) simulations (although larger α values in the midplane can be obtained depending on the disc model, in particular the amplitude of the vertical magnetic field that threads the disc, see, e.g., Simon et al. 2015, 2018). 

        



\chapter{Posters}
        
          \section[Florian Debras: Jupiter and friends] { Jupiter and friends }
\textbf{ Florian Debras}\footnote{\href{mailto: clement.baruteau@free.fr}{clement.baruteau@free.fr} };  Jo le Fort\\
    (1) IRAP; (2)  ENS\\


    
    \emph{poster number: TBA }\\


 
    
\vspace{3 mm}

      have already started to open a gap around their orbit. The particles are uniformly distributed between 52 au and 102 au, so that they approximately all remain between the gaps over the duration of the simulation. This is meant to maximise the particles resolution at the two dust traps from which Clump 1 and Clump 2 originate in our scenario. Fur- thermore, inspired by the dust trapping predictions of Casas- sus et al. (2019, see their Section 3.2.2), we assume that the dust particles have an internal density ρint = 0.1 g cm−3, independent of particles size, instead of a more conventional internal density of a few g cm−3. Our dust particles can therefore be considered as moderately porous particles. This rather low density is overall consistent with the collection by Rosetta of large (>10 μm) porous ag 

        


 
\end{document}